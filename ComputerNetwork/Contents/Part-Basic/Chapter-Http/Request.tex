\section{Http 请求与响应}
\subsection{Http 请求}

常用的 Http 请求有以下几种:
\begin{itemize}
    \item GET:获取数据:是最常见的请求
    \item POST:传输数据:用于传输简单数据
    \item PUT:传输文件
    \item HEAD:获取报文首部,和 GET 类似,但不返回主体
    \item DELETE:删除文件
    \item OPTIONS:询问支持的方法
    \item TRANCE:获取传输路径
    \item CONNECT:用隧道协议连接代理,加密用的
\end{itemize}

\subsubsection*{Cookie}

HTTP 是无状态协议,它不对之前发生过的请求和响应的状态进行管理。Cookie 计数则是用来控制客户端状态的。

客户端向服务端发送请求后,服务端返回的响应中会添加 Cookie 字段用来保存用户状态,之后每一次 CS 之间通讯都会包含 Cookie 的传输与修改。

\subsection{Http 响应}

五种响应状态码如下:

\begin{table}[H]
    \centering
    \caption{状态码类别}
    \label{table:状态码类别}
    \setlength{\tabcolsep}{4mm}
    \begin{tabular}{c|c|cc}
        \toprule
        \textbf{} & \textbf{类别} & \textbf{原因短语} \\
        \midrule
        1XX & informational(信息性状态码) & 请求正在处理 \\
        2XX & Success(成功状态码) & 请求处理完毕 \\
        3XX & Redirection(重定向状态码) & 需要进行附加操作以完成请求 \\
        4XX & Client Error(客户端错误状态码) & 服务器无法处理请求 \\
        5XX & Server Error(服务器错误状态码) & 服务器处理请求出错 \\
        \bottomrule
    \end{tabular}
\end{table}

常用的状态码:

\begin{table}[H]
    \centering
    \caption{常用的状态码}
    \label{table:常用的状态码}
    \setlength{\tabcolsep}{4mm}
    \begin{tabular}{c|c|cc}
        \toprule
        \textbf{类型} & \textbf{状态码} & \textbf{说明} \\
        \midrule
        \multirow{3}{*}{2XX} & 200 & OK,正常处理 \\
         & 204 & No Content,处理成功,但没有资源返回 \\
         & 206 & Partial Content, 返回一部分内容 \\
        \midrule
        \multirow{5}{*}{3XX} & 301 & 永久性重定向,需要更新 URL \\
         & 302 & 临时性重定向 \\
         & 303 & 请求对应的资源存在着另一个 URI,应使用 GET 方法获取 \\
         & 304 & 服务器允许请求访问资源,但未满足条件 \\
         & 307 & 临时重定向 \\
        \midrule
        \multirow{4}{*}{4XX} & 400 & 请求报文存在语法错误 \\
         & 401 & 需要认证,第一次会给出认证提示,第二次表示认证失败 \\
         & 403 & 拒绝访问 \\
         & 404 & 服务器没有对应资源或服务器不愿意给出对应资源 \\
        \multirow{2}{*}{5XX} & 500 & 服务端执行时出现错误 \\
         & 503 & 出无端处于超负荷或停机维护阶段 \\
        \bottomrule
    \end{tabular}
\end{table}

\newpage